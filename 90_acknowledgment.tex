\begin{acknowledgment}

本研究の遂行にあたり,指導教員である増井俊之教授には1年次から温かいご指導をいただきました.本研究のみならず,様々な開発のアイデアに対して学術的あるいは製品化の観点から本質的なご意見をいただき議論させていただきましたことに心から感謝申し上げます.増井研究会では,博士課程の大和比呂志氏,田中優氏,修士課程の左治木隆成氏,同輩の尾崎正和氏,後輩の佐々木雅斗氏には様々なご意見をいただき充実した研究会活動をさせていただきました.卒業生,研究会メンバー諸氏にも多くのご支援をいただきました.

本研究の中心となった,Lyapunov指数を用いた指尖容積脈波の解析は,関西学院大学名誉教授であった故雄山真弓博士にご教授いただいたものです.高校時代に「活用の方法を提案してほしい」と言われた内容が学部でのメインテーマになったことは感慨深いものです.指尖容積脈波の活用について知見とシステムを提供していただいた関西学院高等部の丹羽時彦先生,セレブラルダイナミックス株式会社の海津成男氏に深く御礼申し上げます.

共同研究者である徳島大学デザイン型AI教育研究センターの福井昌則准教授,関西学院大学理工学研究科の萩倉丈氏には日頃から研究含め様々な相談に乗っていただきました.お二人とこれまでに様々な研究を行い,発表してきた経験が生きていると思います.本研究では,関西学院大学経済学部の前田慶士郎氏,関西学院大学文学部総合心理科学科の上田花菜氏,東北大学工学部の村上聡氏に実験計画や実施において多大なご支援をいただき,公私ともにお世話になりました.また,Bridge UI株式会社には研究のバックアップをいただきました.

筆者の研究の原点といえば,関西学院高等部の宮寺良平先生,同中学部の河野隆一先生のご指導だと思います.先生方には在学中から今に至るまで大きな影響を受けました.

本研究は,孫正義育英財団の支援を受けて実施されたものです.ご支援のおかげで自由な研究をさせていただきました.また,財団生諸氏からも良い刺激をいただきました.

公私にわたりご指導,ご支援をいただきました皆様に心から御礼申し上げます.

\begin{flushright}
2022年1月 吉日

佐々木 雄司
\end{flushright}

\begin{figure}[htbp]
    \begin{flushright}
       \includegraphics[width=30mm]{img/signature.jpeg}
    \end{flushright}
\end{figure}

\end{acknowledgment}
