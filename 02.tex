\chapter{関連研究と諸概念の整理}
\label{chap:prevresearch}

\section{ユーザビリティ}

\section{UX}

\subsection{時間相}

\section{満足性}

\section{UXメトリクス}

\section{指尖容積脈波}

ユーザテストでは,パフォーマンスメトリクスや自己申告メトリクス,行動・生理メトリクスを組み合わせて評価することが必要であると前に述べた.パフォーマンスメトリクスでは全ての部分を評価することはできないため,問題がありそうな部分や変更を加えようとする一部分のみを切り出して測定することになる.しかし,この計画立案についてもコストが高くなるため小規模な開発では実施が難しい.