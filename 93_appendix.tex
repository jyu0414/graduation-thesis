\chapter{自由回答}

\section{実験A}

\begin{itemize}
  \item どちらも正解だと思った答えを入力しても、正解にならなかったのは不思議だと思いました。1回目の方は入力するたびに焦りとイラつきがありましたが、2回目の方はずっと不正解でも何度もトライしようという気になりました。
  \item 1回目の実験のとき、何度タップしても反応がなかったので、少し焦った。
  \item 1回目について、まず初めに入力したはずの数字が入力されず、困惑した。そのあと何回かランダムに押せば入力すればいいと気づくも、課題通りに入力したはずなのに不正解となり、始めに聞いた課題を書き間違えたのかと思って焦った。何回か小さい順から入力後、違うアルゴリズムで正解になるパターンだと考え、大きい順から入力するも不正解となり、表示されたまま入力すると正解したので意図を理解した。1回目はその理解までに1分半くらいは使ったと思う。誤入力もありストレスだった。2回目は念のため言われた通りの課題をこなしたが不正解となったので、そのまま入力するのに徹した。タップしたらそのまま数字が入力されるので快適だった。
  \item 1回目ではテンキーの入力判定が悪く、2回目では普段利用している時と変わらない判定であった。
  \item "1回目はボタンの感度が悪く、6回押さないと数字が入力できないときがあったが、2回目は一回押しただけでボタンが反応した.スマホの反応の速さの違いによる感情の変化を調べる実験だったのか?"
  \item 2回目の数字が押しにくかったです。
  \item 素直に反応してくれないとイライラする
  \item 1回目のときは、スムーズに数字を当てたが、2回目のときは、何度押しても反応しない時があったので動揺しました。
  \item クリアボタンの位置が悪魔的
  \item はじめは反応悪いだけかなと思ったけど、ひとつ前の実験から察するに、「そういうやつね」となった。
  \item 1回目に行った時は数字の打ち込みがとても難しく、スマホの画面に問題があるのではないかと思うほど打ち込みにくかった。しかし、2回目はとてもスムーズにいつもスマホを触る時と同じように打ち込むことができた。
  \item 先にスムーズに操作できる回からだったため、2回目の異常さにはすぐに気がつきました。初めは、わざととは思わず、何か実験に不備が出てしまうのではないかと心配になりました。スムーズに操作できないと、押し間違いのミスをしてしまった時のあっ。という気持ちが大きくなり、自分の焦りに繋がるように思いました。2回目の方が、スピードよりも正確性重視で操作を行なっていたように思いますし、小さなストレスも多かったと感じます。
  \item  2回目に数字を入力するのが、うまく行く場合と行かない場合が不規則だったので、最後までなれなかった。
  \item 1回目はスムーズに入力できたのに対して、2回目はとても押し辛かったです。
  \item 2回目の方が反応がよかった。1回目の方は何回も数字を押さないと反応しなかった。
  \item タップした時に自分が思ったように数字を押せていなかったら焦ってミスが生まれた。また、何回も繰り返していくと最初より精度が落ちた気がした。
  \item 2度目はテンキーの反応が悪く苛立ちがあったため、クリア・決定の操作も誤ることがあった
  \item 2回目の方が入力しづらかった。
  \item 1回目の何度も押さないといけない方は、煩わしかった。
  \item 2回目の時に一度数字を選択しても反映されなくて、連打しているうちに2回押しちゃってまた1番最初からやり直しということが何回かあって、少し焦った。
  \item 1度目は何度か押しても反応しない場合があった2度目はスムーズに行けた
  \item 1回目は数字のボタンが押しにくく何回も入力し直した。2回目はスムーズに入力できストレスフリーで行うことができた。
  \item 1回目に数字を押した時は反応しづらかった.直前のアンケートでは普通に反応していたため、何かそういうアプリかなあと思いました2回目は普通に打てたので楽しかったです
  \item 初めの方は動きが悪くて、やりづらいなと思いました。
  \item 1回目の実験で入力する際に、押しても反応しない時があったのがとてもいらいらしました。2回目は押しやすかったです。
  \item 一回目の実験の操作は、数字ボタンが反応しないことがあり難しかった。しかし、クリアと決定ボタンには不具合がないのが不思議だった。二回目の実験の操作は正常に反応したため快適に感じた。
  \item 2回目の実験の際に、ボタンの反応が悪くなり、入力ミスを起こしてしまった。
  \item 1回目は一度入力すれば文字が反映されたのに対し、2回目はランダムで1〜5回タップしないと文字が反映されなかった
  \item バラバラに数字を並べられると意外と難しい。2回目はわざと一発で入力できないようにされているのだなと感じた。自分のめんどくさがりで短気なところから何回もタップして同じ数字を2回入力してしまったりした。単純にゲーム感覚で面白かった。
  \item 1回目の実験ではレスポンスが早く感じたが2回目の実験ではレスポンス以前の問題で数回押してもなぜか反応しないことがあったため意図的に反応させないとしたら結構悪用が可能なのではと考えている。
  \item 2回目では画面の反応が悪くストレスを感じた。また、なぜか問題解答の正確さも落ちてしまった。
  \item 1回目は普通だったけれど、2回目は数回に1回だけ反応していた。
  \item 1回目は数字が打ちづらかった。
  \item 一度目は反応が鈍く反映されにくく、二度目は普通に押せば反映されるように感じた
  \item 入力のしやすさに差があった。
  \item 2回目の実験は入力しづらかった。
  \item 1回目の時は少し反応が鈍かったので、2回目の時の方が多く正解することが出来ました。
  \item 2回目にストレスがかかるように設計されたゲームであると気づいた。反応が悪いほうをするときの方が慎重になってゲームしていた気がする。反応しないと思って連打するときもあったためミスも増えたように感じた。
  \item 1回目は何度かタップしないと文字入力ができなかった。2回目はスムーズにできた。
  \item 初め、文字が打ちにくくて機械の不具合かと思った。
  \item 6つの数字を、3つずつに分けて考えると素早く判断することができると感じた。何回か繰り返して慣れてくると、逆に入力ミスや小さい数字の見落としなどのミスが増えてきたように感じた。
  \item 数字のボタンを押しても反応しないことがある。
  \item ・1.2回目共、数字のキーをなぞるように打っていくとミスが少ないと感じ、そのように行った。・1回目と比較して2回目では数字を打つのが難しく、その分ミスタッチが多くなったため慎重に行う方がいいと判断し、そうした
  \item 2回目は1回目と違って、テンキーが反応しない時があった。

\end{itemize}

\section{実験B}

\begin{itemize}
  \item 1回目の方は時間内に1回だけしかクリア出来なかったので、ゲーム性があって面白かった。2回目の方はスムーズに操作できたが、何をしてもクリアしたので、ゲーム感覚よりかは作業のように感じた。
  \item 2回目の実験のとき、ボールが思ったように動かなかった。半分以上進んだときにやり直しになったときは、少し悲しかった。
  \item 数字の入力を経験した後なので1回目のボールの実験も面倒なやつだろうなと思って取り組んだ。最後まで行けなかったのが悔しい。2回目はとても快適だった。
  \item 1回目ではボールを動かす際の移動距離はランダムであるような気がした。2回目では指の動きとボールの動きがリンクしており、ゴールが容易であった。また、2回目でははみ出した判定が線に対してボールの1番離れた所となっていた。
  \item 1回目は指でなぞった通りにボールが動いたが、2回目はボールの動きを操れなかった。スマホの反応の速さの違いによる感情の違いを調べる実験なのか?
  \item 一回目が楽だったので、二回目はめんどくさくなると予測していたが想像以上だったので、かなりのストレスだった。
  \item 1度目のものは何がだめで振り出しに戻されてるのかわからなかったです。2回目もなぜ戻されているのかわかりませんでしたが、スムーズに動く分やりやすかったと思います。
  \item 2回目反応が悪く球を動かしにくかった
  \item ひたすら同じことを繰り返すと飽きてくる。
  \item 1回目のときは、なかなかボールが動かなくて困りましたが、2回目のときはスムーズに動くようになったので一瞬さっきの動かし方が悪かったのかと思いました。
  \item 2回目が特に難しい
  \item はじめは反応悪いだけかなと思ったけど、時間が経っても改善されないので「そういつやつね」となった。1回目が順調だっただけに、2回目が一回も成功できずモヤモヤした。
  \item 数字を打ち込む作業と同じくとてもやりづらく、指先を繰り返し小刻みになぞる動作だった。とてももどかしく感じた。2回目はなぞることがとても楽で、あまり神経質にならずにスピードを速めながらなぞることができた。
  \item 先に、上手く操作ができない回からおこなったため、この感度が正常なのかと無理やり思い込んで頑張ろうとしていたように思います。そのため、うまく操作できない自分への焦りを感じました。身近な機器がスムーズに操作ができることは、勉強や自由時間のクオリティを格段に上げてくれていることを実感したように思います。
  \item 1回目は、あまりにもタッチの反応が悪かったので、最後までうまく行く方法がわからなかった。2回目は1回目と比べて驚くほど反応が甘かった。
  \item 一回目全くボールが動かなくて、最初の坂で止まってしまった。2回目はスムーズでやりやすかった。
  \item 1回目の方が2回目よりも反応が良かった。しかし、どちらも数字なら実験に比べると反応が悪かった気がした。
  \item こちらも、最初の方はすぐにゴール出来たが、繰り返していくうちに思った位置までボールがきていないことが増えミスが増えた。一回ミスすると繰り返してしまった
  \item 2回目は反応が悪く思った通りに動かなかったがむしろ集中できた
  \item 2回目の方が反応が悪く、1回もクリアできなくて残念だった。
  \item 携帯が、ちゃんと思い通りに動いているのは、嬉しい。感謝の気持ちが芽生えた。
  \item 1回目は一度もミスすることなく何回も成功できたが、2回目は全然指に合わせてボールが動かなくて少しイライラした。一度は成功できたものの、他はすぐに一番最初からになったので悔しかった。
  \item 1度目は進むスピードが遅く、少しのズレで最初からのやり直しだったため、難しかった
  \item 1回目はスムーズにボールを動かすことができ、はみ出してもまたやり直せばいいという考えから素早く操作できたが、2回目は動かしにくかったため慎重に動かしすぎて時間内にゴールまで辿り着けなかった。
  \item 数字の方と交互にやったため、2回目は触りやすくなってるのかなと寧ろ3分間あの単調な作業をする方が頭おかしくなりそうです
  \item 初めの方は一筆ではかけず、2回目はスイスイかくことができました。
  \item 2回目の実験で、ボールをうまく進められなかったり、なんとかして進めようとしたが最初に戻ったりして、とてもイライラしました。1回目はスムーズにできました。
  \item 一回目の実験の操作は、なぞった際に進む速度がバラバラで難しかった。また、途中でボールの上に指を合わせる必要がないことに気づいた。二回目の実験の操作は、ボールの上をなぞった通りに動いたため快適だった。一回目よりもアウト判定がゆるいように感じた。
  \item 1回目の実験ではボールの反応が悪かったものの、2回目では回復したため、少し驚いた。
  \item 1回目はスムーズにすらすらとボールが動いたのに対し、2回目は徐々にしか動かず、1回目よりも2回目の方が「線から外れた判定」が厳しいと感じた
  \item ボールをスライドさせるという説明文では指をボールに置くのか進行方向に指を動かすのかわからなかった。一回目は非常にカクカクしていてやりづらかった。これは数字と反対でわざと一回目がやりづらくされていたと気づいた。一回目がカーブのところでコースアウトしてしまったのでスムーズに動く二回目でも慎重になった。
  \item 一回目は明らかにレスポンスが悪く実験どころではない状況であった。ゲームであれをやられたらレビューが悲惨なことになること間違いなしである。2回目は大凡のレスポンスが改善されておりあの動作ならまだ問題ないと思う。
  \item 2回目では思った通りにボールが動かずより正確になぞれるように注意した。一回しかうまくいかなかったがうまく行った時は達成感を感じた。
  \item 一回目は普通だったけれど、2回目は滑りづらくて、私の手が人間じゃなくなったのかとおもった。そういう実験なのかなぁどっちなのかなあと思っていたら2回目の数字もへんだったからそういう実験なのだなと思った。
  \item 1回目はボールが動かしづらかった。
  \item 一度目は有効な範囲が狭く、触っても反応しているかわからないようなほどしか進まなかったが、二度目はさらさら出来たため、どこまではみ出して動かしてもいいか試したりしながら進めた。だいぶはまだしても大丈夫だったように感じた。
  \item 2回目の実験が動かし易かった。
  \item 1回目の実験はなぞりにくかった。
  \item 1回目はボールが少しずつしか進まなかったので、2回目の方がスムーズに行うことができました。
  \item 1回目の方がボールの反応が悪く自分が思った通りに動かなくストレスがあった。2回目と比べした後の疲労感という意味でも1回目の方がしんどかった。1回目は1回もクリアできていなかったため、最後になればなるほど焦りが出てきてまたミスも増えたのかもしれない
  \item ボールが1回目に比べてとても動かしにくかった。少しでもずれると最初に戻ってしまうので難しかった。
  \item 二回目はボールが動きづらくなっていたため、非常にやりづらく、多少の苛立ちがあった。
  \item コツを掴むと素早くなぞることができたが、一度変な癖がついてしまって何回か連続してミスしてしまった。途中から綺麗になぞれてるのかよく分からない感じがした。
  \item 指でなぞっている方向にボールが移動しなかった。スムーズに移動しなかったためやりにくかった。
  \item 数字の実験と同様に、1回目と比較して2回目はなぞりにくかった。そのため、1回目は一筆書きであまり道からずれないことを考えずに行っていたが、2回目は何度もタップし直し、道からそれないように慎重になった。その分1回ゴールした際の達成度が高かった。
  \item 2回目は指を動かしてもボールがついてこなかった。1回目の曲がり角まで行くとスタート位置に戻って、ボールが一瞬赤くなった。
\end{itemize}
