
\begin{jabstract}

使いやすいシステムを設計するには,実ユーザに対するユーザテストによってユーザの利用時の品質を示すUXを評価し,製品の改善に繋げるサイクルが重要である.しかし,UX評価を行っているソフトウェア開発企業は少なく,行っていたとしても評価した結果が製品の改善に繋がっているケースがほとんど無いと指摘されている.ユーザテストの導入ハードルを下げ普及させるために,専門的な知識が無くても容易にユーザテストを行いシステムの問題点を発見できるようにする必要がある.そこで本研究では,(1)容易なユーザテストを可能にするための新たなUX指標として自律神経バランス(ANB)と最大リヤプノフ指数(LLE)を提案し,(2)それらの指標を用いたUX評価システムを開発する.

(1)テストユーザの生理的反応に基づく評価指標は確立された手法が無いため普及していないものの,生理的反応からは機械的に解釈可能なデータを得られるため容易なユーザテスト手法になり得る.そこで,容積脈波のカオス揺らぎからストレスを測定する技術を応用した新たなUX指標を提案し有用性を明らかにした.

UXの概念では時間変化を考慮することが重要である.これまでに黒須のUXグラフなどUXの時間変化を記述する方法が提案されてきたが,使用前,使用後,習熟後といった数週間から数ヶ月レベルの期間で大まかに示すものだった.そこで,(2)UX評価システムでは,シーケンシャルにデータを取得できる生理的反応を指標として採用することで,システム使用中の数分から数時間程度のより詳細なUX変化を記録し可視化することを可能にする.

\end{jabstract}


\begin{eabstract}

In order to design a system that is easy to use, it is important to conduct user tests on actual users to evaluate UX, which indicates the quality of the system when used by the user, and to use this cycle to improve the product. However, it is pointed out that few software development companies conduct UX evaluations, and even if they do, there are few cases where the evaluation results lead to product improvements. In order to reduce the hurdle of user testing, it is necessary to make it easy for users to conduct user testing and find problems in the system without specialized knowledge. In this study, we (1) propose Autonomic Nervous Balance (ANB) and Largest Lyapunov Exponent (LLE) as new UX indicators to enable easy user testing, and (2) develop a UX evaluation system using these indicators.

(1) Although evaluation indices based on the physiological responses of test users are not widely used because there is no established method, physiological responses can provide mechanically interpretable data, which can be an easy user testing method. However, physiological responses can provide mechanically interpretable data that can be easily used as a user testing method. We proposed a new UX index based on the technique of measuring stress from chaotic fluctuations of volumetric pulse waves, and clarified its usefulness.

In the concept of UX, it is important to consider time variation. Although some methods have been proposed to describe changes in UX over time, such as Kurosu's UX graphs, they have been used to roughly indicate the time period from a few weeks to a few months, such as before use, after use, and after mastery. Therefore, the (2) UX evaluation system uses physiological responses, which can be obtained sequentially, as an indicator to record and visualize more detailed UX changes over a period of several minutes to several hours during system use.

\end{eabstract}
