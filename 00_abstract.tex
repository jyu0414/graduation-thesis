
\begin{jabstract}

ソフトウェア開発では,機能性よりもむしろ使いやすさや体験を広告する製品も見られ,ユーザの利用時の品質を示すUXの向上がビジネスに直結する課題となっている.使いやすいシステムを設計するには,実ユーザに対するユーザテストによってUXを評価し,製品の改善に繋げるサイクルが重要である.しかし,UX評価を行っているソフトウェア開発企業は少なく,行っていたとしても評価した結果が製品の改善に繋がっているケースがほとんど無いと指摘されている.ユーザテストの導入ハードルを下げ普及させるために,専門的な知識が無くても容易にユーザテストを行いシステムの問題点を発見できるようにするシステムの開発が必要である.そこで本研究では,(1)容易なユーザテストを可能にするための新たなUX指標として自律神経バランス(ANB)と最大リヤプノフ指数(LLE)を提案し,(2)それらの指標を用いたUX評価システムを開発する.

(1)テストユーザの生理的反応に基づく評価指標は確立された手法が無いため普及していないものの,生理的反応は機械的に解釈可能なデータを得ることで容易なユーザテスト手法になり得る.そこで,容積脈波のカオス揺らぎからストレスを測定する技術を応用し,新たなUX指標を提案し有用性を明らかにした.

UXの概念では時間変化を考慮することが重要である.これまでに黒須のUXグラフなどUXの時間変化を記述する方法が提案されてきたが,使用前,使用後,習熟後までの数週間から数ヶ月レベルの期間で大まかに示すものだった.そこで,(2)UX評価システムでは,シーケンシャルにデータを取得できる生理的反応を指標として採用することで,システム使用中の数分から数時間程度のより詳細なUX変化を記録し可視化することを可能にする.

\end{jabstract}


\begin{eabstract}

In software development, as some software advertises the user experience rather than functionality, improving UX has become a business-critical issue. In order to design a user-friendly system, UX evaluation through user testing and reflecting it on the products is important. However, few software development companies conduct UX evaluations, and even if they do, the results are rarely linked to product improvements. To popularize user testing, it is necessary to develop a system that allows users to easy user testing and find out the UI problems without technical knowledge. In this study, we propose (1) Autonomic Nervous Balance (ANB) and Largest Lyapunov Exponent (LLE) as new UX indicators for easy user testing, and (2) develop a UX evaluation system using these indicators.

(1) The evaluation index based on physiological reactions is not common because there is no established method. However, physiological responses can be an easy method for user testing by obtaining mechanically interpretable data. Therefore, we propose a new UX index by applying the technique of measuring stress from chaotic fluctuations of volumetric pulse waves.

The time span is the key concept of UX. Some methods have been proposed to describe the change of UX such as Kurosu's UX graph but it shows the time span roughly. Therefore, (2) the UX evaluation system uses physiological response as an indicator, which can obtain data sequentially, and enables to record and visualize more detailed UX changes from a few minutes to a few hours during system use.

\end{eabstract}
